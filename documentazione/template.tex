\documentclass[12pt,a4paper,titlepage]{article}
%va sempre messo article per "program documentation"
\usepackage[italian]{babel}
\usepackage[T1]{fontenc}
\usepackage[latin1]{inputenc}
\usepackage{titlesec}
\usepackage{hyperref}
\usepackage[a4paper,top=2cm,bottom=2cm,left=1cm,right=1cm]{geometry}
\usepackage{soulutf8,color}

\usepackage{emptypage}                     
% pagine vuote senza testatina e piede di pagina

%\usepackage{hyperref}                     
% collegamenti ipertestuali

\usepackage{fancyhdr}
% pacchetto per intestazione e pie pagina

\pagestyle{fancy}


% \\ indica interruzione di riga

% compilate 2 volte per documenti con indice

% {\em qui testo in corsivo}
% {\bfseries qui testo in grossetto}

%LISTE NUMERATE
%\begin{enumerate}
%\item primo
%\item secondo
%\item terzo
%\end{enumerate}

%LISTE PUNTATE
%\begin{itemize}
%\item primo
%\item secondo
%\item \dots
%\end{itemize}

%TABELLA
%\begin{tabular}{|c|c|c|}
%indica una tabella con 3 colonne e pos. testo centrale. La barra verticale ( | ) indica che vi e' una linea divisoria verticale tra le celle.
%\hline	linea separatrice orizzontale
%testo1& testo2& testo3\\
% & segna la fine del testo nella cella , \\ indica il fine della riga 

%GRAFICI
%\begin{figure}
%\includegraphics{filegrafico}
%comando per includere le immagini (controllare i formati)
%\caption{didascalia}
%\label{nome}
%\end{figure}


\lhead{Nome Gruppo}
\chead{}
\rhead{Nome Capitolo}
\lfoot{Nome Documento}
\cfoot{}
\rfoot{\thepage}
\renewcommand{\headrulewidth}{0.4pt}
\renewcommand{\footrulewidth}{0.4pt}

\begin{document} 

\title{Documento}
\author{SWEg Group}
\date{}
\maketitle

\lhead{SWEg Group}
\chead{}
\lfoot{Analisi dei Requisiti}
\cfoot{}
\rfoot{\thepage}
\renewcommand{\headrulewidth}{0.2pt}
\renewcommand{\footrulewidth}{0.2pt}
\rhead{Registro Modifiche}

\section{Registro Modifiche}
\small %rippicciolisce il testo
\begin{tabular}{|l|c|c|c|}
\hline
{\textbf{Modifica}}&{\textbf{Nome}}&{\textbf{Data}}&{\textbf{Ver.}}\\
\end{tabular}

\tableofcontents
%crea indice automaticamente
\thispagestyle{empty}

\newpage



\section{Titolo del paragrafo}
\subsection{Titolo del sottoparagrafo}
Testo Testo Testo.\\
Altro Testo Altro Testo.\\
\\
Testo distanziato. Testo distanziato. 

\section{Titolo del paragrafo 2}
\subsection{Titolo del sottoparagrafo 2}
\subsubsection{Titolo del paragrafino}
Testo Testo Testo 2.\\
Altro Testo Altro Testo 2.\\
\\
\subsubsection{Titolo paragrafino 2}
Testo distanziato 2. Testo distanziato 2. 



\end{document}