\section{Titolo}{
	
	%PER LE TABELLE DI TRACCIAMENTO VEDI ANALISI REQUISITI E PIANO DI PROGETTO CHE HANNO MOLTE TABELLE
	Testo:
	\begin{itemize}\itemsep1pt
		\item lista;
		\item lista.
	\end{itemize} 
	
	\subsection{SottoTitolo}{
		Testo.
		
		\begin{figure}[ht]
			%serie figure
			\begin{subfigure}[b]{0.5\textwidth}
				\includegraphics[width=\textwidth]{\docsImg seriefigura1.png}
				%\vspace{-40pt}
				\caption{Titolo Figura 1}
				\label{Titolo Figura 1}
			\end{subfigure}
			\begin{subfigure}[b]{0.5\textwidth}
				\includegraphics[width=\textwidth]{\docsImg seriefigura2.png}
				%\vspace{-40pt}
				\caption{Titolo Figura 2}
				\label{Titolo Figura 2}
			\end{subfigure}
			\\
			\\
			\begin{subfigure}[b]{0.5\textwidth}
				\includegraphics[width=\textwidth]{\docsImg seriefigura3.png}
				%\vspace{-40pt}
				\caption{Titolo Figura 3}
				\label{Titolo Figura 3}
			\end{subfigure}
		\end{figure}
	}
	\subsection{SottoTitolo}{
		Testo.
		\\
		Testo.
		\begin{figure}[ht]
			\centering
			\includegraphics{\docsImg figura.png}
			\caption{Testo figura}
			\label{Testo figura}
		\end{figure}
		
		Testo.
		\\
		Testo:
		\begin{itemize}\itemsep1pt
			\item lista; con:
			\begin{itemize}\itemsep1pt
				\item sottolista
				\item \textbf{testogrossetto}: da da da;
				\item \textit{Testo corsivo}.\\ 
				Ancora testo.
				\item testo.
			\end{itemize}
			\item Continua la lista di prima.
		\end{itemize}
	}
	\subsection{Sottotitolo}{
		Testo.
		\subsubsection{SottoSottoTitolo}{
			Testo \url{http://www.url.com/}.
		}
	}
}
