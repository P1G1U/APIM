\section{Tecnologie Utilizzate}{
	In questa sezione illustreremo le tecnologie e i linguaggi utilizzati nel progetto, indicando pro e contro delle scelte utilizzate.
	\subsection{Linguaggi}{
		\subsubsection{HTML 5}{
			HTML\textsubscript{g} è la particella elementare di Internet: il linguaggio di markup che dà vita ai siti Web statici. HTML5 è la versione 5 di questo linguaggio. HTML5 apre le porte a una nuova serie di funzioni per contenuti interattivi e animati che funzionano universalmente su qualsiasi piattaforma o tipo di dispositivo.
			\textbf{Vantaggi :}
			\begin{itemize}\itemsep1pt
				\item Funziona su la maggior parte dei computer e sui dispositivi mobili;
				\item Video e animazioni supportati senza plug-in\textsubscript{g} esterni;
				\item Pensiero indirizzato sempre più al web semantico e HTML5 ne è portavoce;
				\item Maggior flessibilità e potenzialità rispetto alle precedenti versioni.
			\end{itemize}
			\textbf{Svantaggi :}
			\begin{itemize}\itemsep1pt
				\item Visualizzazione non uniforme sulle versioni precedenti dei browser o su Internet Explorer\textsubscript{g};
				\item Strumenti non completamente sviluppati. C'è bisogno di un linguaggio di supporto per le pagine dinamiche.
			\end{itemize}
		}
		\subsubsection{CSS3}{
			Il CSS\textsubscript{g} è un linguaggio con il quale formattare le pagine Web. Un file CSS\textsubscript{g} viene normalemente chiamato un foglio di stile, e va associato ad una o più pagine Web. I fogli di stile nel progetto saranno rigorosamente esterni. CSS3 aggiorna le funzionalità e le componenti stilistiche.
			\textbf{Vantaggi :}
			\begin{itemize}\itemsep1pt
				\item Separata la struttura del sito dalla presentazione;
				\item Più facile progettazione di accessibilità;
				\item Template\textsubscript{g} unico per varie pagine senza ripetizione;
				\item Facile la modifica in caso di cambiamento;
				\item Grafica accattivante per gli utenti.
			\end{itemize}
			\textbf{Svantaggi :}
			\begin{itemize}\itemsep1pt
				\item I browser più datati hanno una non corretta interpretazione dei CSS\textsubscript{g};
				\item Maggiore attenzione sulla psicologia di marketing grafico posta verso l'utente.
			\end{itemize}
		}
		\subsubsection{Javascript}{
			La caratteristica principale di Javascript\textsubscript{g}, è quella di essere un linguaggio di scripting. Ci permetterà di eseguire particolari operazioni grazie alla flessibilità di questo linguaggio orientato agli oggetti ed eventi. Tali funzioni di script possono essere opportunamente inserite in file HTML\textsubscript{g}, in pagine JSP o in appositi file separati con estensione .js poi richiamati nella logica di business.
			\textbf{Vantaggi :}
			\begin{itemize}\itemsep1pt
				\item Possibilità di rendere dinamiche le pagine web e di estendere funzionalità;
				\item Il linguaggio di scripting è più sicuro ed affidabile perché in chiaro e da interpretare, quindi può essere filtrato;
				\item Gli script hanno limitate capacità, per ragioni di sicurezza, per cui non è possibile fare tutto con Javascript\textsubscript{g}, ma occorre abbinarlo ad altri linguaggi evoluti, ( come Jolie\textsubscript{g} );
				\item Il codice Javascript\textsubscript{g} viene eseguito sul client per cui il server non è sollecitato più del dovuto e la velocità dell'applicazione complessiva è migliore;
			\end{itemize}
			\textbf{Svantaggi :}
			\begin{itemize}\itemsep1pt
				\item Il codice è visibile e può essere letto da chiunque;
				\item La mancanza di tipizzazione del linguaggio potrebbe indurre a commettere errori nel codice e rendere più difficile la progettazione dei test.
			\end{itemize}
		}
		\subsubsection{Jolie}{
			Jolie\textsubscript{g} fissa i concetti di programmazione di microservizi come funzionalità del linguaggio native: gli elementi di base del software non sono oggetti o funzioni, ma piuttosto servizi che possono sempre essere trasferiti e replicati in base alle esigenze. Distribuzione e riusabilità si raggiungono con la semplice progettazione e codifica.\\
			\textbf{Vantaggi :}
			\begin{itemize}\itemsep1pt
				\item Linguaggio orientato agli oggetti, basato su Java, con tutti i vantaggi dei linguaggi ad oggetti;
				\item Linguaggio nato appositamente per i microservizi che è punto focale del nostro progetto;
				\item Funziona perfettamente sia in locale sia in remoto, il codice non altera la logica dei programmi;
				\item I servizi possono scambiare dati utilizzando diversi protocolli, non vi è uno specifico e possono essere diversi d'entrata che in uscita;
				\item Essendo un codice orientato ai microservizi ogni prodotto creato può essere riutilizzato;
				\item È dotato nativamente di primitive per workflow, questo rende il codice fluido per le esigenze, evitando le variabili computazionali soggette a errori per verificare ciò che è accaduto finora in un calcolo;
				\item Jolie\textsubscript{g} è dotato di una solida semantica per la gestione di errori della programmazione parallela. Possiamo  aggiornare il comportamento dei gestori degli errori in fase di esecuzione;
				\item Implementa Leonardo\textsubscript{g}: servizio Jolie\textsubscript{g} che agisce come un server web in grado di interagire con le applicazioni web scritte in diverse tecnologie (JSON, XML, AJAX, GWT). 
			\end{itemize}
			\textbf{Svantaggi :}
			\begin{itemize}\itemsep1pt
				\item Non compatibile con tutti i linguaggi e ancora in fase di prototipazione l'interazione con database non relazionali;
				\item Linguaggio nuovo e non usato in ogni suo ramo e sfaccettatura, manca infatti documentazione completa ed esaustiva.
			\end{itemize}
		}
		\subsubsection{SQL}{
			SQL è il linguaggio che andremo ad usare per quanto riguarda la parte di database della nostra applicazione web. Jolie\textsubscript{g} offre dei comandi semplici e si interfacciano comodamente con il linguaggio per basi di dati relazionali.\\
			È un linguaggio standardizzato per database basati sul modello relazionale (RDBMS) progettato per:
			\begin{itemize}
				\item creare e modificare schemi di database (DDL - Data Definition Language);
				\item inserire, modificare e gestire dati memorizzati (DML - Data Manipulation Language);
				\item interrogare i dati memorizzati (DQL - Data Query Language);
				\item creare e gestire strumenti di controllo ed accesso ai dati (DCL - Data Control Language).
			\end{itemize}
			Nonostante il nome, non si tratta dunque solo di un semplice linguaggio di interrogazione, ma alcuni suoi sottoinsiemi si occupano della creazione, della gestione e dell'amministrazione del database.\\
			\textbf{Vantaggi :}
			\begin{itemize}\itemsep1pt
				\item Già implementato e studiato per il nostro linguaggio cardine Jolie\textsubscript{g};
				\item Già a conoscenza della totalità del team;
				\item Elastico e integrato da tempo nelle applicazione web;
				\item Molto veloce e permette di gestire un alto numero di operazioni/secondo;
				\item Se ben programmato in principio avvantaggia maggiormente la lettura, importante per l'applicazione web;
				
			\end{itemize}
			\textbf{Svantaggi :}
			\begin{itemize}\itemsep1pt
				\item Gestisce operazioni non troppo complicate;
				\item Limitato su basi di dati troppo grandi;
				\item Difficile riadattamento nel caso di modifica della struttura del database.
			\end{itemize}
		}
	}
	\subsection{Frameworks}{
		\subsubsection{AngularJS}{
			{AngularJS}\textsubscript{g} è un framework JavaScript\textsubscript{g} per lo sviluppo di applicazioni Web client side. Pur essendo relativamente giovane (la versione 1.0 è stata rilasciata nel 2012), il progetto ha riscosso un notevole successo dovuto all'approccio di sviluppo proposto e all'infrastruttura fornita che incoraggia l’organizzazione del codice e la separazione dei compiti nei vari componenti.\\
			Per raggiungere questo obiettivo, AngularJS\textsubscript{g} da un lato esalta e potenzia l’approccio dichiarativo del HTML\textsubscript{g} nella definizione dell'interfaccia grafica, dall'altro fornisce strumenti per la costruzione di un’architettura modulare e testabile della logica applicativa di un’applicazione.\\
			AngularJS\textsubscript{g} fornisce tutto quanto occorre per creare applicazioni moderne che sfruttano le più recenti tecnologie, come ad esempio le Single Page Application, cioè applicazioni le cui risorse vengono caricate dinamicamente su richiesta, senza necessità di ricaricare l’intera pagina. Tra le principali funzionalità a supporto dello sviluppo di tali applicazioni segnaliamo:
			\begin{itemize}
				\item il binding bidirezionale (\textit{two-way binding})
				\item la dependency injection
				\item il supporto al pattern MVC
				\item il supporto ai moduli
				\item la separazione delle competenze
				\item la testabilità del codice
				\item la riusabilità dei componenti
			\end{itemize}
			\textbf{Vantaggi :}
			\begin{itemize}\itemsep1pt
				\item Estremamente espressivo, leggibile e di facile implementazione. Un'evoluzione all'HTML\textsubscript{g} per facilitare le applicazioni web;
				\item È completamente estensibile e funziona bene con altre librerie. Ogni funzione può essere modificato o sostituito in base alle esigenze del flusso di lavoro di sviluppo;
				\item Integrazione perfetta con uno dei pattern scelti (che nei prossimi capitoli illustriamo) Model View ViewModel;
				\item Data-Binding\textsubscript{g} Bidirezionale di AngularJS\textsubscript{g} gestisce la sincronizzazione tra il DOM e il modello, e viceversa;
				\item AngularJS\textsubscript{g} ha un sottosistema integrato di Independence Injection\textsubscript{g} che aiuta lo sviluppatore a creare un'applicazione facile da sviluppare, capire e provare; 
				\item Utilizza le direttive\textsubscript{g}, cioè possono essere usate per creare tag HTML\textsubscript{g} personalizzati che funzionano come nuovi widget personali. Possono anche essere usati per "decorare" elementi con un comportamento e manipolare attributi DOM in un modo interessante.
			\end{itemize}
			\textbf{Svantaggi :}
			\begin{itemize}\itemsep1pt
				\item Framework\textsubscript{g} che si deve usare completamente e non si può lasciare spazio all'incomprensione;
				\item Non vi sono IDE specifici o dedicati gratuiti;
				\item Non vi sono delle linee guida internazionali, ma solo spunti vari nel web.
			\end{itemize}
		}
		\subsubsection{Spring Framework - Spring Boot}{
			Il progetto Spring Boot permette di semplificare, e di molto, lo sviluppo di applicazioni basate sul framework Spring.\\
			La Spring Framework è un framework applicativo e inverte i controlli del contenitore per la piattaforma Java. Le funzionalità di base del framework possono essere utilizzate da qualsiasi applicazione Java, ma ci sono delle estensioni per la creazione di applicazioni web. 
			\textbf{Vantaggi :}
			\begin{itemize}\itemsep1pt
				\item Pur essendo molto ampio, grazie alla sua modularità si può scegliere di integrare solo alcune parti all'interno del progetto;
				\item Open Source\textsubscript{g};
				\item Basato su piattaforma Java, già conosciuta dei membri del gruppo;
				\item Adatto per le RESTful web service framework;
				\item Semplifica le sviluppo delle applicazioni basate su Spring;
				\item Risolve il problema di quali librerie Spring utilizzare e quale versione;
				\item Risolve il problema dell'individuazione e configurazione di tutti i bean che saranno gestiti dal framework e necessari alla nostra applicazione. 
			\end{itemize}
			\textbf{Svantaggi :}
			\begin{itemize}\itemsep1pt
				\item Poca documentazione ufficiale (il sito contiene una sola pagina funzionante!);
				\item Poca o nulla conoscenza da parte del gruppo.
			\end{itemize}
	}
	
	}
	\subsection{Librerie}{
		\subsubsection{Leonardo: il web server di Jolie}{
			Leonardo\textsubscript{g} è un server web sviluppato unicamente in Jolie\textsubscript{g}. È molto flessibile e può scalare da un semplice contenuto statico HTML\textsubscript{g} fino a sostenere un complesso servizio web dinamico.\\
			\textbf{Vantaggi :}
			\begin{itemize}\itemsep1pt
				\item Scritto interamente in Jolie\textsubscript{g} e facilmente implementabile nel prodotto;
				\item Si interfaccia con HTML\textsubscript{g}, JQuery;
				\item Permette l'uso dei Cookies\textsubscript{g}.	
			\end{itemize}
			\textbf{Svantaggi :}
			\begin{itemize}\itemsep1pt
				\item Non è estendibile con qualsiasi linguaggio anche se vi sono già molti prototipi.
			\end{itemize}
		}
		\subsubsection{Highcharts}{
			È una libreria JavaScript\textsubscript{g} che permette di gestire e inserire grafici all'interno della nostra applicazione web. La utilizziamo in particolare per mostrare le statistiche generiche del sito agli amministratori e i dati di interesse per i microservizi offerti dagli utenti. \\
			\textbf{Vantaggi :}
			\begin{itemize}\itemsep1pt
				\item Compatibilità con tutti i browser moderni, sia desktop che mobile;
				\item Interfacciamento semplice con AngularJS\textsubscript{g} grazie a delle direttive;
				\item Aggiornamento dei grafici in tempo reale;
				\item Possibilità di esportare i grafici in vari formati;
				\item Open source\textsubscript{g}, quindi personalizzabile. E gratuito per fini non commerciali.
			\end{itemize}
			\textbf{Svantaggi :}
			\begin{itemize}\itemsep1pt
				\item Non è compatibile con vecchie versioni di AngularJS\textsubscript{g};
				\item Non è mai stata utilizzata dai membri del gruppo.
			\end{itemize}
		}
		\subsubsection{Angular Material}{
			Angular Material è l’implementazione del Material Design\textsubscript{g} in AngularJS\textsubscript{g}. Fornisce un insieme di componenti per l’interfaccia utente riutilizzabili, testati e accessibili, basati sul Material Design\textsubscript{g}. \\
			\textbf{Vantaggi :}
			\begin{itemize}\itemsep1pt
				\item Compatibilità con tutti i browser moderni, sia desktop che mobile;
				\item Facile relazionarsi con AngularJS\textsubscript{g} essendo la sua implementazione;
				\item Documentazione\textsubscript{g} e informazioni più volte usate e prodotte, oltre che esempi specifici.
			\end{itemize}
			\textbf{Svantaggi :}
			\begin{itemize}\itemsep1pt
				\item Anche questa tecnologia come AngularJS\textsubscript{g} non è stata usata all'interno del gruppo e richiede particolare attenzione;
				\item Tecnologia non definitiva e in continuo aggiornamento.
			\end{itemize}
		}
		\subsubsection{Highcharts-ng}{
			È una direttiva AngularJS\textsubscript{g} per Highcharts. Servirà nella pratica ad usare le due librerie nel framework scelto.\\
			\textbf{Vantaggi :}
			\begin{itemize}\itemsep1pt
				\item Ausilio integrale della libreria Highcharts;
				\item La documentazione presenta eventi ed è possibile comunicare direttamente con l'ideatore.
			\end{itemize}
			\textbf{Svantaggi :}
			\begin{itemize}\itemsep1pt
				\item È una tecnologia in continuo sviluppo migliorata dagli utenti, ha possibili variazioni e aggiornamenti.
			\end{itemize}
		}
	}
}