\section{Architettura Database}{
	In questa sezione viene descritta la progettazione del database che conterrà tutte le informazioni utili al nostro sistema. Abbiamo scelto di utilizzare un database relazionale.
	
	\subsection{Progettazione concettuale}{
		Nella progettazione concettuale viene modellata la realtà da rappresentare. Sono state individuate le classi e le relazioni tra di esse, definendo così la struttura che avrà il database.
		
		\begin{figure}[ht]
				\includegraphics[width=\textwidth]{\docsImg ProgettazioneConcettuale.png}
				\caption{Progettazione Concettuale}
				\label{Fig. Progettazione Concettuale}
		\end{figure}
			
			\subsubsection{Schema concettuale}
			
			\subsubsection{Classi}
			\textit{User}\\
			Questa é l'entità che rappresenta tutti gli utenti che utilizzeranno il software, siano essi utenti normali o amministratori. Questa entità come si vede dallo schema é superclasse di Utenti e di Admin. Gli attributi sono: 
			\begin{center}
			\begin{tabular}{lcc}
				\textbf{Attributo}&\textbf{Tipo}&\textbf {Vincolo}\\
				\hline
				Username&String&PRIMARY KEY\\
				Password&String&null\\
				Nome&String&null \\
				Cognome&String&null \\
				DataNascita&Date&null\\
				Mail&String&null \\
				NumeroCarta&String&null \\
				Indirizzo&String&null \\
				Telefono&String&null \\
				FotoProfilo&Mediumblob&null\\
				Bio&String&null \\
				SaldoCoin&Double&null \\
				IsAdmin&Bool&null\\
			\end{tabular}
			\end{center}
		
			\textit{Utenti}\\
			Questa entità rappresenta gli utenti normali che utilizzeranno la piattaforma. E' una sottoclasse di \textbf{Users} e quindi eredita i suoi attributi. In particolare il campo \textbf{Bio} contiene una piccola biografia con valutazione dell'utente.  \\
			\textit{Admin}\\
			Questa entità invece rappresenta gli amministratori che gestiranno la piattaforma. Come la precedente é una sottoclasse di \textbf{Users} e ne eredita gli attributi. Per gli admin è stato aggiunto il campo \textbf{IsAdmin} per verificare quando un utente che accede è un amministratore. \\
			\textit{API}\\
			Questa entità rappresenta le API presenti nel sistema. Gli attributi sono:
			\begin{center}
			\begin{tabular}{lcc}
				\hline
				\textbf{Attributo}&\textbf{Tipo}&\textbf{Vincolo}\\ \hline
				Nome API&String&PRIMARY KEY\\
				LinkFile&String&null\\
				LinkPdf&String&null \\
				NumeroVoti&Int&null \\
				RatingMedio&Double&null \\
				TempoMedioRisposta&double&null \\
				TotaleChiamate&Int&null \\
				Traffico&Double&null \\
				TempoRispostaTotale&Double&null \\
			\end{tabular}
			\end{center}
			In particolare in questa tabella sono contenute le statistiche di ogni API come il \textbf{Tempo Medio Risposta}, il \textbf{Traffico} di dati che comporta l'tilizzo di quel microservizio, il numero \textbf{Totale Chiamate} e il \textbf{Tempo Risposta Totale}. \\
			\textit{Key}\\
			Questa entità rappresenta le chiavi che verranno assegnate ad un utente quando acquista un microservizio per potervi accedere ed utilizzarlo. Gli attributi sono: \\
			\begin{center}
			\begin{tabular}{lcc}
				\textbf{Attributo}&\textbf{Tipo}&\textbf{Vincolo}\\ \hline
				Key&String&PRIMARY KEY\\
				DataScadenza&Date&null \\
				MaxByte&Double&null \\
				TempoUso&Double&null \\
			\end{tabular}
			\end{center}
			In particolare gli attributi \textbf{DataScadenza, MaxByte e TempoUso} sono i limiti entro i quali la chiave può essere utilizzata. Secondo la policy di acquisto di un microservizio, al raggiungimento di tale limite la chiave scadrà e non permetterà più l'utilizzo di tale microservizio.
			\textit{Commenti}\\
			Questa entità rappresenta i commenti ed il voto  che gli utenti possono lasciare su ogni microservizio che hanno utilizzato. Gli attributi sono:
			\begin{center}
			\begin{tabular}{lcc}
				\textbf{Attributo}&\textbf{Tipo}&\textbf{Vincolo}\\ \hline
				IdCommento&Int&PRIMARY KEY\\
				Testo&String&null \\
				Rating&Int&null \\
			\end{tabular}
			\end{center}
			In particolare textbf{Rating} sarà il giudizio che l'utente lascerà del microservizio utilizzato. \\
			
			\textit{Categorie}
			Questa entità rappresenta la suddivisione in categorie dei vari microservizi. L'amministratore può aggiungere o togliere categorie a propria scelta. Gli attributi sono: 
			\begin{center}
			\begin{tabular}{lcc}
				\textbf{Attributo}&\textbf{Tipo}&\textbf{Vincolo}\\ \hline
				IdCategoria&Int&PRIMARY KEY\\
				Categoria&String&null \\			
			\end{tabular}
			\end{center}
			
			\textit{Acquisti}
			Questa entità contiene lo storico degli acquisti effettuati dagli utenti.
			\begin{center}
			\begin{tabular}{lcc}
				\textbf{Attributo}&\textbf{Tipo}&\textbf{Vincolo}\\ \hline
				IdAcquisto&Int&PRIMARY KEY\\
			\end{tabular}
			\end{center}
	\subsubsection{Associazioni}
		\begin{itemize}
			\item\textit{User-Acquisto}: molteplicitàn - 1. Un utente può acquistare uno o più microservizi, un microservizio può esser acquistata da uno o più utenti.
			\item\textbf{User-API}: molteplicità 1 - n. Un utente può mettere a disposizione uno o più microservizi, mentre un microservizio é inserito nel database dal un solo utente.
			\item\textbf{User-Commenti}: molteplicità 1 - n. Un utente può lasciare diversi commenti in diverse API, mentre un commento é lasciato per forza da un solo utente;
			\item\textbf{API-Commenti}: molteplicità 1 - n. Una API può avere diversi commenti, ma un commento può essere lasciato su una sola API.	
			\item\textbf{Acquisto-Key}: molteplicità 1 - 1. Ogni acquisto avrà associata la propria chiave. 
			\item\textbf{API - Key}: molteplicità 1 - n. Una API avrà ad essa associate più chiavi, mentre una chiave appartiene ad una sola API.
			\item\textbf{API-Categorire}: molteplicità 1 - n. Ad una categoria appartengono una o più API, mentre una API appartiene ad una sola categoria.
			\item\textbf{Users-Key}: molteplicità 1 -n. Un utente può possedere una o più chiavi, mentre una chiave sarà associata ad un solo utente.
		\end{itemize}				
	}
	\subsection{Progettazione logico-relazionale}{
		La progettazione logico-relazionale segue la progettazione concettuale. Da qui le classi o entità verranno chiamate relazioni. In questa fase vengono inserite in ogni relazione le chiavi esterne per rappresentare le gerarchie e le associazioni.
		\subsubsection{Schema logico-relazionale}
		\begin{figure}[ht]
			\centering
			\includegraphics[width=\textwidth]{\docsImg ProgettazioneLogica}
			\caption{Progettazione Logica}
			\label{Fig. Progettazione Logica}
		\end{figure}
		\subsubsection{Gerarchie}
		\textbf{User - Utenti - Admin}: La gerarchia é stat implementata come tabella unica. La due sottoclassi \textbf{Utenti} e \textbf{Admin} sono comprese nella superclasse \textbf{Users}. Viene solamente aggiunto un attributo discriminante per distinguere un utente normale da un amministratore.\\
		\subsubsection{Relazioni}
		\textbf{Categorie}\\
		Gli attributi sono: \\
			\begin{center}
			\begin{tabular}{lcc}
				\textbf{Attributo}&\textbf{Tipo}&\textbf{Vincolo}\\ \hline
				IdCategoria&Int&PRIMARY KEY\\
				Categoria&Varchar&null \\
			\end{tabular}
			\end{center}
			
		\textbf{Users}\\
			Relazione creata dall'implementazione tramite tabella unica della gerarchia User - Utenti - Admin. Gli attributi sono:
			\begin{center}
			\begin{tabular}{lcc}
				\textbf{Attributo}&\textbf{Tipo}&\textbf{Vincolo}\\ \hline
				Username&Varchar&PRIMARY KEY\\
				Password&Varchar&null \\				
				Nome&Varchar&null \\
				Cognome&Varchar&null \\
				DataNascita&Date&null \\				
				Mail&Varchar&null \\
				NumeroCarta&Varchar&null \\
				Indirizzo&Varchar&null \\
				Telefono&Varchar&null \\
				Bio&Varchar&null \\
				FotoProfilo&Mediumblob&null\\
				SaldoCoin&Double&null \\
				IsAdmin&Boolean&null \\
				NomeAPI&Varchar&FOREIGN KEY\\
			\end{tabular}
			\end{center}				
			
		\textbf{API}\\
			Gli attributi sono:
			\begin{center}
			\begin{tabular}{lcc}
				\textbf{Attributo}&\textbf{Tipo}&\textbf{Vincolo}\\ \hline
				NomeAPI&Varchar&PRIMARYKEY\\
				LinkFile&Varchar&null \\
				LinkPdf&Varchar&null \\
				NumeroVoti&Int&null \\
				RatingMedio&Double&null \\
				TempoMedioRisposta&Double&null \\
				TotaleChiamate&Int&null \\
				Traffico&Doule&null \\
				TempoRispostaTotale&Double&null \\
				Username&Varchar&FOREING KEY\\
				Categoria&Varchar&FOREIGN KEY\\
			\end{tabular}
			\end{center}
			
		\textbf{Commenti}\\
			Gli attributi sono:
			\begin{center}
			\begin{tabular}{lcc}
				\textbf{Attributo}&\textbf{Tipo}&\textbf{Vincolo}\\ \hline
				IdCommento&Int&PRIMARY KEY\\
				Testo&Varchar&null \\
				Rating&Int&null \\
				Username&Varchar&FOREIGN KEY\\
				NomeAPI&Varchar&FOREIGN KEY\\
			\end{tabular}
			\end{center}
		
		\textbf{Key}
			Questa tabella si crea dalla relazione tra Users e API. Rappresenta l'acquisto da parte di un utente di un microservizio. All'utente viene assegnata una chiave con la quale poter accedere al microservizio. Gli attributi sono:
			\begin{center}
			\begin{tabular}{lcc}
				\textbf{Attributo}&\textbf{Tipo}&\textbf{Vincolo}\\ \hline
				Key&Varchar&PRIMARY KEY\\
				DataScadenza&Date&null \\
				MaxByte&Double&null \\
				TempoUso&Time&null \\
				Username&Varchar&FOREIGN KEY\\
				NomeAPI&Varchar&FOREIGNKEY\\			
			\end{tabular}
			\end{center}
			In perticolare, la \textbf{DataScadenza} indica la data fino alla quale é stato acquistato il microservizio; similmente \textbf{MaxByte} e \textbf{TempoUso} stanno ad indicare rispettivamente il traffico massimo ed il tempo massimo con i quali si può usufruire del microservizio in base al contratto d'acquisto stipulato.\\
			
		\textbf{Acquisti}\\
		Gli attributi sono: \\
		\begin{center}
		\begin{tabular}{lcc}
			\textbf{Attributo}&\textbf{Tipo}&\textbf{Vincolo}\\ \hline
			IdAcquisto&Int&PRIMARY KEY\\
			Username&Varchar&FOREIGN KEY\\
			Key&Varchar&FOREIGN KEY\\
		\end{tabular}
		\end{center}
			
		\textbf{Crea}
			Questa tabella viene a crearsi dalla relazione tra Users e API. Rappresenta il caricamento nel sistema di un microservizio da parte di un utente. Gli attributi sono:
			\begin{center}
			\begin{tabular}{lcc}
				\textbf{Attributo}&\textbf{Tipo}&\textbf{Vincolo}\\ \hline
				IdMS&Int&PRIMARY KEY\\
				NomeAPI&Varchar&FOREIGN KEY\\
				Username&Varchar&FOREIGN KEY\\			
			\end{tabular}
			\end{center}
			
		}
	
}