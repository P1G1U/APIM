\section{Architettura Database}{
	In questa sezione viene descritta la progettazione del database che conterra' tutte le informazioni utili al nostro sistema. Abbiamo scelto di utilizzare un database relazionale.
	
	\subsection{Progettazione concettuale}{
		Nella progettazione concettuale viene modellata la realta' da rappresentare. Sono state individuate le classi e le relazioni tra di esse, definendo cosi' la struttura che avra' il database.
		
		\begin{figure}[ht]
			%serie figure
			\begin{subfigure}[b]{0.5\textwidth}
				\includegraphics{"ProgettazioneConcettuale"}
				\caption{Progettazione Concettuale}
				\label{Fig. Progettazione Concettuale}
			\end{subfigure}
			
			\subsubsection{Schema concettuale}
			
			\subsubsection{Classi}
			\textit{User}\\
			Questa e' l'entita' che rappresenta tutti gli utenti che utilizzeranno il software, siano essi utenti normali o amministratori. Questa entita' come si vede dallo schema e' superclase di Utenti e di Admin. Gli attributi sono: 
			\begin{center}
			\begin{tabular}{lcc}
				\textbf{Attributo}&\textbf{Tipo}&\textbf {Vincolo}\\
				\hline
				Username&String&PRIMARY KEY\\
				Password&String&null\\
				Nome&String&null \\
				Cognome&String&null \\
				DataNascita&Date&null\\
				Mail&String&null \\
				NumeroCarta&String&null \\
				Indirizzo&String&null \\
				Telefono&String&null \\
				FotoProfilo&Mediumblob&null\\
				Bio&String&null \\
				SaldoCoin&Double&null \\
				IsAdmin&Bool&null\\
			\end{tabular}
			\end{center} \\
			\textit{Utenti}\\
			Questa entita' rappresenta gli utenti normali che utilizzeranno la piattaforma. E' una sottoclasse di \textbf{Users} e quindi eredita i suoi attributi. In particolare il campo \textbf{Bio} contiene una piccola biografia con valutazione dell'utente.  \\
			\textit{Admin}\\
			Questa entita' invece rappresenta gli amministratori che gestiranno la piattaforma. Come la precedente e' una sottocalsse di textbf{Users} e ne eredita gli attributi. Per gli admin è stato aggiunto il campo \textb{IsAdmin} per verificare quando un utente che accede è un amministratore. \\
			\textit{API}\\
			Questa entita' rappresenta le API presenti nel sistema. Gli attributi sono:
			\begin{center}
			\begin{tabular}{lccc}
				&Attributo&Tipo&Vincolo\\
				Nome API&String&PRIMARY KEY\\
				LinkFile&String&null\\
				LinkPdf&String&null \\
				NumeroVoti&Int&null \\
				RatingMedio&Double&null \\
				TempoMedioRisposta&double&null \\
				TotaleChiamate&Int&null \\
				Traffico&Double&null \\
				TempoRispostaTotale&Double&null \\
			\end{tabular}
			\end{center}
			In particolare in questa tabella sono contenute le statistiche di ogni API come il \textbf{Tempo Medio Risposta}, il \textbf{Traffico} di dati che comporta l'tilizzo di quel microservizio, il numero \textbf{Totale Chiamate} e il \textbf{Tempo Risposta Totale}. \\
			\textit{Key}\\
			Questa entita' rappresenta le chiavi che verranno assegnate ad un utente quando acquista un microservizio per potervi accedere ed utilizzarlo. Gli attributi sono: \\
			\begin{center}
			\begin{tabular}{lccc}
				&Attributo&Tipo&Vincolo\\
				Key&String&PRIMARY KEY\\
				DataScadenza&Date&null \\
				MaxByte&Double&null \\
				TempoUso&Double&null \\
			\end{tabular}
			\end{center}
			In particolare gli attributi \textbf{DataScadenza, MaxByte e TempoUso} sono i limiti entro i quali la chiave puo' essere utilizzata. Secondo la policy di acquisto di un microservizio, al raggiungimento di tale limite la chiave scadra' e non permettera' piu' l'utilizzo di tale microservizio.
			\textit{Commenti}\\
			Questa entita' rappresenta i commenti ed il voto  che gli utenti possono lasciare su ogni microservizio che hanno utilizzato. Gli attributi sono:
			\begin{center}
			\begin{tabular}{lccc}
				&Attributo&Tipo&Vincolo\\
				IdCommento&Int&PRIMARY KEY\\
				Testo&String&null \\
				Rating&Int&null \\
			\end{tabular}
			\end{center}
			In particolare textbf{Rating} sara' il giudizio che l'utente lascera' del microservizio utilizzato. \\
			
			\textit{Categorie}
			Questa entita' rappresenta la suddivisione in categorie dei vari microservizi. L'amministratore puo' aggiungere o togliere categorie a propria scelta. Gli attributi sono: 
			\begin{center}
			\begin{tabular}{lccc}
				&Attributo&Tipo&Vincolo\\
				IdCategoria&Int&PRIMARY KEY\\
				Categoria&String&null \\			
			\end{tabular}
			\end{center}\\
			
			\textit{Acquisti}
			Questa entita' contiene lo storico degli acquisti effettuati dagli utenti.
			\begin{center}
			\begin{tabular}{lccc}
				&Attributo&Tipo&Vincolo\\
				IdAcquisto&Int&PRIMARY KEY\\
			\end{tabular}
			\end{center}
	\subsubsection{Associazioni}
		\begin{itemize}
			\item\textit{User-Acquisto}: molteplicita'n - 1. Un utente puo' acquistare uno o piu' microservizi, un microservizio puo' esser acquistata da uno o piu' utenti.
			\item\textbf{User-API}: molteplicita' 1 - n. Un utente puo' mettere a disposizione uno o piu' microservizi, mentre un microservizio e' inserito nel database dal un solo utente.
			\item\textbf{User-Commenti}: molteplicita' 1 - n. Un utente puo' lasciare diversi commenti in diverse API, mentre un commento e' lasciato per forza da un solo utente;
			\item\textbf{API-Commenti}: molteplicita' 1 - n. Una API puo' avere diversi commenti, ma un commento puo' essere lasciato su una sola API.	
			\item\textbf{Acquisto-Key}: molteplicita' 1 - 1. Ogni acquisto avra' associata la propria chiave. 
			\item\textbf{API - Key}: molteplicita' 1 - n. Una API avra' ad essa associate piu' chiavi, mentre una chiave appartiene ad una sola API.
			\item\textbf{API-Categorire}: molteplicita' 1 - n. Ad una categoria appartengono una o piu' API, mentre una API appartiene ad una sola categoria.
			\item\textbf{Users-Key}: molteplicita' 1 -n. Un utente puo' possedere una o piu' chiavi, mentre una chiave sara' associata ad un solo utente.
		\end{itemize}				
								
			\\
			\\
	}
	\subsection{Progettazione logico-relazionale}{
		La progettazione logico-relazionale segue la progettazione concettuale. Da qui le classi o entita' verranno chiamate relazioni. In questa fase vengono inserite in ogni relazione le chiavi esterne per rappresentare le gerarchie e le associazioni.
		\subsubsection{Schema logico-relazionale}
		\begin{figure}[ht]
			\centering
			\includegraphics{"ProgettazioneLogica"}
			\caption{Progettazione Logica}
			\label{Fig. Progettazione Logica}
		\end{figure} \\
		\subsubsection{Gerarchie}
		\textbf{User - Utenti - Admin}: La gerarchia e' stat implementata come tabella unica. La due sottoclassi \textbf{Utenti} e \textbf{Admin} sono comprese nella superclasse \textbf{Users}. Viene solamente aggiunto un attributo discriminante per distinguere un utente normale da un amministratore.\\
		\subsubsection{Relazioni}
		\textbf{Categorie}\\
		Gli attributi sono: \\
			\begin{center}
			\begin{tabular}{lccc}
				&Attributo&Tipo&Vincolo\\
				IdCategoria&Int&PRIMARY KEY\\
				Categoria&Varchar&null \\
			\end{tabular}
			\end{center}\\
			
		\textbf{Users}\\
			Relazione creata dall'implementazione tramite tabella unica della gerarchia User - Utenti - Admin. Gli attributi sono:
			\begin{center}
			\begin{tabular}{lccc}
				&Attributo&Tipo&Vincolo\\
				Username&Varchar&PRIMARY KEY\\
				Password&Varchar&null \\				
				Nome&Varchar&null \\
				Cognome&Varchar&null \\
				DataNascita&Date&null \\				
				Mail&Varchar&null \\
				NumeroCarta&Varchar&null \\
				Indirizzo&Varchar&null \\
				Telefono&Varchar&null \\
				Bio&Varchar&null \\
				FotoProfilo&Mediumblob&null\\
				SaldoCoin&Double&null \\
				IsAdmin&Boolean&null \\
				NomeAPI&Varchar&FOREIGN KEY\\
			\end{tabular}
			\end{center} \\					
			
		\textbf{API}\\
			Gli attributi sono:
			\begin{center}
			\begin{tabular}{lccc}
				&Attributo&Tipo&Vincolo\\
				NomeAPI&Varchar&PRIMARYKEY\\
				LinkFile&Varchar&null \\
				LinkPdf&Varchar&null \\
				NumeroVoti&Int&null \\
				RatingMedio&Double&null \\
				TempoMedioRisposta&Double&null \\
				TotaleChiamate&Int&null \\
				Traffico&Doule&null \\
				TempoRispostaTotale&Double&null \\
				Username&Varchar&FOREING KEY\\
				Categoria&Varchar&FOREIGN KEY\\
			\end{tabular}
			\end{center}\\
			
		\textbf{Commenti}\\
			Gli attributi sono:
			\begin{center}
			\begin{tabular}{lccc}
				&Attributo&Tipo&Vincolo\\
				IdCommento&Int&PRIMARY KEY\\
				Testo&Varchar&null \\
				Rating&Int&null \\
				Username&Varchar&FOREIGN KEY\\
				NomeAPI&Varchar&FOREIGN KEY\\
			\end{tabular}
			\end{center}\\
		
		\textbf{Key}
			Questa tabella si crea dalla relazione tra Users e API. Rappresenta l'acquisto da parte di un utente di un microservizio. All'utente viene assegnata una chiave con la quale poter accedere al microservizio. Gli attributi sono:
			\begin{center}
			\begin{tabular}{lccc}
				&Atttributo&Tipo&Vincolo\\
				Key&Varchar&PRIMARY KEY\\
				DataScadenza&Date&null \\
				MaxByte&Double&null \\
				TempoUso&Time&null \\
				Username&Varchar&FOREIGN KEY\\
				NomeAPI&Varchar&FOREIGNKEY\\			
			\end{tabular}
			\end{center}\\
			In perticolare, la \textbf{DataScadenza} indica la data fino alla quale e' stato acquistato il microservizio; similmente \textbf{MaxByte} e \textbf{TempoUso} stanno ad indicare rispettivamente il traffico massimo ed il tempo massimo con i quali si puo' usufruire del microservizio in base al contratto d'acquisto stipulato.\\
			
		\textbf{Acquisti}\\
		Gli attributi sono: \\
		\begin{center}
		\begin{tabular}{lccc}
			&Attributo&Tipo&Vincolo\\
			IdAcquisto&Int&PRIMARY KEY\\
			Username&Varchar&FOREIGN KEY\\
			Key&Varchar&FOREIGN KEY\\
		\end{tabular}
		\end{center}\\
			
		\textbf{Crea}
			Questa tabella viene a crearsi dalla relazione tra Users e API. Rappresenta il caricamento nel sistema di un microservizio da parte di un utente. Gli attributi sono:
			\begin{center}
			\begin{tabular}{lccc}
				&Attributo&Tipo&Vincolo\\
				IdMS&Int&PRIMARY KEY\\
				NomeAPI&Varchar&FOREIGN KEY\\
				Username&Varchar&FOREIGN KEY\\			
			\end{tabular}
			\end{center}
			
		}
	
}