\section{MockUp}{
	Il mockup è l'attività di riprodurre un oggetto o modello in scala ridotta o maggiorata. In questa sezione infatti riportiamo e descriviamo tre delle principali visioni della nostra applicazione web. L'idea è di presentare, oltre alla prima pagina, due delle importanti funzioni del nostro prodotto: la pagina del profilo utente e la ricerca di un nuovo microservizio. Evitando le pagine di inserimento di un nuovo microservizio e statistica del nostro prodotto che presenteranno rispettivamente una lista di form per l'inserimento del ms e una serie di grafici non ancora opportunamente studiati.\\
	Per la nostra visione grafica sono stati scelti espressamente due modelli molto importanti, già compresi e assimilati dai consumatori finali:
	\begin{itemize}
		\item Store di Apple (Formato Desktop);
		\item GitHub website.
	\end{itemize}
	Le motivazioni delle scelte sono che nel primo caso il prodotto ha alle spalle innumerevoli designer ed esperti del settore. Ha una visione commerciale che invita il consumatore all'acquisto. I servizi sono offerti in modo chiaro per categorie e ci fornisce uno stimolo creativo su cui prendere spunto pulito ed elegante.\\
	Nel secondo caso la scelta è stata riposta per la maggior parte delle nostra visione. È intanto un servizio offerto per programmatori da programmatori, quindi pensiamo che i requisiti coincidano: cioè una distribuzione di servizi per programmatori che offrono e cercano nuovi servizi per i loro obbiettivi. Ha una buonissima sezione sociale, niente di invasivo come grandi piattaforme moderne, ma abbastanza per rimanere in contatto con le sezioni di interesse. Anche qui ci sono più categorie e più livelli di ricerca e di immagazzinamento. I prodotti che vengono caricati sono simili alla nostra visione di microservizi (avranno una documentazione, un interfaccia, un numero di dati di dettaglio). Infine la pagina principale è molto simile all'idea che voglia far avere al supporter dei microservizi.\\ 
	Ora in dettaglio mostriamo le scelte fatte per le abbiamo fatto di design dell'interfaccia.
	
	\subsection{Home}{
		La home è la prima pagina che ci si trova dopo la corretta fase di login. \\
		\begin{figure}[ht]
			\centering
			\includegraphics{\docsImg PrototipoHome.png}
			\caption{Idea di visione della Home}
			\label{PrototipoHome}
		\end{figure}
		La prima pagina offre un menù con le principali operazioni che l'utente finale può eseguire. Notiamo:
		\begin{itemize}
			\item Una ordine dei microservizi per \textit{Categoria}. In un futuro rigoglioso per la produzione di microservizi, colui che usufruirà del servizio deve avere visione specifica e caratteristica di quello che vuole per comporre la sua API. Una differenziazione intelligente faciliterà la progettazione del API finale e quindi il cliente tornerà soddisfatto della prestazione d'opera.
			\item \textit{Profilo} è una pagina che andremo a presentare dopo. Messa apposta in seguito alle categorie perché è secondario dall'idea che voglia imprimere al consumatore.
			\item La sezione \textit{Carica}, che abbiamo deciso di non rappresentare con un prototipo è l'idea del caricamento di un nuovo microserivio. Perché vogliamo far si che oltre all'acquisto vi sia un'importante valore di popolazione del prodotto finale. La pagina conterrà, come già spiegato, una serie di form che permette al nostro attore di caricare il suo personale microservizio.
			\item Infine la parte di \textit{Aggiornamenti} non è ancora stata ideata alla perfezione , ma l'idea è di dare una visione dell'andamento dei microservizi collegati all'account oltre che un piccolo reportage del portafoglio modificato nel tempo.
		\end{itemize}
		Il menù viene inteso come header del sito e si ritroverà in ogni pagina con la variante che nella sezione attuale il link sarà in grossetto. E non mancheranno la possibilità di cercare grazie alla form e un comodo set di pulsanti "avanti" e "indietro".\\
		Subito sotto l'header troviamo nella parte sinistra un titolo semplice con descrizione. Accanto dei link veloci ed utili per l'utente oltre che l'importante importo disponibile nell'account.\\
		Terza e ultima parte della nostra Home è una serie di importanti microservizi che gestiamo noi personalmente nella prima fase del progetto. L'idea è che un algoritmo auto genera la home con i microservizi che più sono attinenti con le ricerche del account, o più fiorenti equilibrando durata da quando sono stati prodotti e rating generico. Ma sono ancora solamente idee.\\
		Sulla destra invece link utili sui microservizi di ordine generale, che possono variare da domande frequenti a vere e proprio guide alla creazione del microservizio.
	}
	
	\subsection{Profilo}{
		La seguente pagina che vogliamo presentare è la sezione profilo. Come da standard il link presenta il nome in grassetto \\
		\begin{figure}[ht]
			\centering
			\includegraphics{\docsImg PrototipoProfilo.png}
			\caption{Idea di visione della Pagina profilo}
			\label{PrototipoProfilo}
		\end{figure}
		La prima pagina offre un menù con le principali operazioni che l'utente finale può eseguire. Notiamo oltre al principale "header" che non è cambiato due sezioni della pagina:
		\begin{itemize}
			\item 
		\end{itemize}	
	}		
}
