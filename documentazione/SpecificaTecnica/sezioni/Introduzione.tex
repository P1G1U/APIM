\section{Introduzione}{
	\subsection{Scopo del Documento}
		Lo scopo generale del documento è di misurare l'efficienza e tenerla in considerazione preventivamente. Importantissimo per il committente che tiene d'occhio la stima delle risorse. \\
		\`E in particolare una dichiarazione di interfaccia di pianificazione e consuntivazione. Sempre redatto dal \textit{Project Manager} schematizzato:
		\begin{enumerate}
			\item Definizione degli obbiettivi;
			\item Analisi dei rischi;
			\item Descrizione del modello di processo di sviluppo;
			\item Suddivisione di sottoinsiemi;
			\item Attività di progetto;
			\item Stima dei costi;
			\item Consuntivo attività.
		\end{enumerate} 

	\subsection{Glossario}
		Al fine di evitare ambiguità e ottimizzare la comprensione dei documenti, viene incluso un Glossario, nel quale saranno inseriti i termini tecnici, acronimi e parole che necessitano di essere chiarite.\\
		Un glossario è una raccolta di termini di un ambito specifico e circoscritto. In questo caso per raccogliere termini desueti e specialistici inerenti al progetto. 
		
	\subsection{Riferimenti}
		\subsubsection{Normativi}
			\begin{itemize}
				\item \textbf{Vincoli organigramma e dettagli tecnico-economici}: \\
				\textcolor{blue}{\url{http://www.math.unipd.it/~tullio/IS-1/2016/Progetto/PD01b.html}}.
				\item \textbf{Norme di Progetto}: \\
				"Norme di Progetto v1.0.0".
			\end{itemize}
		\subsubsection{Informativi}
			\begin{itemize}
				\item \textbf{Metriche di Progetto}: \\
				\textcolor{blue}{\url{https://it.wikipedia.org/wiki/Metriche_di_progetto}}.
				\item \textbf{Modello incrementale}: \\
				\textcolor{blue}{\url{https://it.wikipedia.org/wiki/Modello_incrementale}}.
				\item \textbf{Modello incrementale}: \\
				\textcolor{blue}{\url{https://it.wikipedia.org/wiki/Metodologia_agile}}.
				\item \textbf{Gestione di progetto}: \\
				\textcolor{blue}{\url{http://www.math.unipd.it/~tullio/IS-1/2016/Dispense/L04.pdf}}.
			\end{itemize}
}
