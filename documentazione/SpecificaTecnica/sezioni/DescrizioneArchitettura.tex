\section{Descrizione Architettura}{
	
	\subsection{Metodo di specifica}{
	
	    Il metodo scelto per esporre l'architettura é tramite un approccio top-down, il che vuol dire che verrá prima presentata una architettura molto astratta e poi verrá passo a passo espansa fino ad arrivare 
	    ad un livello molto basilare dove potranno essere identificate le singole classi e le loro sottoclassi, queste ultime verranno trattate nello specifico all'interno della fase della Progettazzione in dettaglio. 
	    Partiremo quindi ad analizzare il rapporto che é presente tra i vari package per poi espandere i package ed analizzare nello specifico da cosa sono formati, che lavoro svolgono e come comunicano tra di loro,
	    quindi passeremo ad analizzare le classi che ne fanno parte, ovvero i metodi che contengono, il loro funzionamento e l'obiettivo per cui sono state sviluppate. Verranno poi mostrati dei design pattern utilizzati
	    e mostrati alcuni esempi degli stessi nell'appendice A.	Il Proponente\textsubscript{g} ha chiesto esplicitamente che venisse utillizzato all'interno del progetto il linguaggio Jolie\textsubscript{g}, un linguaggio da lui sviluppato,
	    e ha messo a disposizione anche Leonardo\textsubscript{g}, percui nella fase di progettazzione il team ha dovuto integrare questa tecnologia al resto del progetto.
		
	}
	\subsection{Architettura generale}{
		Il Proponente\textsubscript{g} nel presentare il capitolato ha riportato un'architettura di massima che rappresenterà come dovrà essere il prodotto finale. Tale architettura è la seguente:
	
		\begin{figure}[H]
			\centering
			\includegraphics[width=\textwidth]{\docsImg Architettura.png}
			\caption{Architettura di massima del progetto}
			\label{Architettura di massima del progetto}
		\end{figure}
		
		L'APIMarket\textsubscript{g} dovrà permettere ad un utente di caricare una API\textsubscript{g}, deve poter acquistare APIKey\textsubscript{g} che gli permettano di utilizzare API\textsubscript{g} già presenti nell'APIMarket\textsubscript{g} e infine avere la possibilità di monitorare i dati riferiti alle API\textsubscript{g}.
		L'architettura sopra riportata rappresenta una possibile soluzione di come deve essere il prodotto finale: attraverso un APIGateway\textsubscript{g} è possibile controllare che le API\textsubscript{g} caricate siano conformi a 
		standard che verranno concordati con il Proponente\textsubscript{g}, sempre l'APIGateway\textsubscript{g} si occuperà di gestire acquisto e la validazione dell'APIKey\textsubscript{g} in possesso dell'utente (o che desidera acquistare), infine 
		sarà sempre lui a monitorare le chiamate alle API\textsubscript{g} e a fornire i dati necessari richiesti.
		Per interfacciare l'utente all'APIGateway\textsubscript{g} è necessaria un'interfaccia. Per definire l'interfaccia il team ha optato per un'architettura MVVM (Model-View-ViewModel) che si discosta dall'architettura MVC
		in quanto in questa architettura esiste un collegamento bilaterale tra la \textit{View} e la \textit{Model} e questo comporta che ogni modifica che viene effettuata sull'una, va a modificare anche i parametri dell'altra.
		La scelta dell'utilizzo di questa architettura nei confronti di una architettura MVC viene trattata nel capitolo riguardante il Front-end\textsubscript{g}.
		Il Proponente\textsubscript{g} ha espressamente chiesto che il team usufruisse di una tecnologia da lui creata, ovvero Jolie\textsubscript{g}, perciò il team ha deciso di sviluppare il Back-end\textsubscript{g} secondo un'architettura a microservizi così da poter 
		sfruttare a pieno le potenzialità di questa tecnologia. 
		
}
