\section{Descrizione Architettura}{
	
	\subsection{Metodo di specifica}{
	
	    Il metodo scelto per esporre l'architettura é tramite un approccio top-down, il che vuol dire che verrá prima presentata una architettura molto astratta e poi verrá passo a passo espansa fino ad arrivare 
	    ad un livello molto basilare dove potranno essere identificate le singole classi e le loro sottoclassi, queste ultime verranno trattate nello specifico all'interno della fase della Progettazzione in dettaglio. 
	    Partiremo quindi ad analizzare il rapporto che é presente tra i vari package per poi espandere i package ed analizzare nello specifico da cosa sono formati, che lavoro svolgono e come comunicano tra di loro,
	    quindi passeremo ad analizzare le classi che ne fanno parte, ovvero i metodi che contengono, il loro funzionamento e l'obiettivo per cui sono state sviluppate. Verranno poi mostrati dei design pattern utilizzati
	    e mostrati alcuni esempi degli stessi nell'appendice A.	Il Proponente ha chiesto esplicitamente che venisse utillizzato all'intrno del progetto il linguaggio Jolie\textsubscript{g}, un linguaggio da lui sviluppato,
	    e ha messo a disposizione anche Leonardo\textsubscript{g}, percui nella fase di progettazzione il team ha dovuto integrare questa tecnologia al resto del progetto.
		
		\begin{figure}[ht]
			%serie figure
			\begin{subfigure}[b]{0.5\textwidth}
				\includegraphics[width=\textwidth]{\docsImg seriefigura1.png}
				%\vspace{-40pt}
				\caption{Titolo Figura 1}
				\label{Titolo Figura 1}
			\end{subfigure}
			\begin{subfigure}[b]{0.5\textwidth}
				\includegraphics[width=\textwidth]{\docsImg seriefigura2.png}
				%\vspace{-40pt}
				\caption{Titolo Figura 2}
				\label{Titolo Figura 2}
			\end{subfigure}
			\\
			\\
			\begin{subfigure}[b]{0.5\textwidth}
				\includegraphics[width=\textwidth]{\docsImg seriefigura3.png}
				%\vspace{-40pt}
				\caption{Titolo Figura 3}
				\label{Titolo Figura 3}
			\end{subfigure}
		\end{figure}
	}
	\subsection{Architettura generale}{
		Testo.
	\\
		Testo.
		\begin{figure}[ht]
			\centering
			\includegraphics{\docsImg figura.png}
			\caption{Testo figura}
			\label{Testo figura}
		\end{figure}
		
		Testo.
		\\
		Testo:
		\begin{itemize}\itemsep1pt
			\item lista; con:
			\begin{itemize}\itemsep1pt
				\item sottolista
				\item \textbf{testogrossetto}: da da da;
				\item \textit{Testo corsivo}.\\ 
				Ancora testo.
				\item testo.
			\end{itemize}
			\item Continua la lista di prima.
		\end{itemize}
		}
	\subsection{Sottotitolo}{
		Testo.
		\subsubsection{SottoSottoTitolo}{
			Testo \url{http://www.url.com/}.
		}
	}
}
