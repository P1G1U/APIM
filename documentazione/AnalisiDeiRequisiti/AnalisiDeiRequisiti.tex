% !TeX spellcheck = it_IT
\documentclass[12pt,a4paper,titlepage]{article}
%va sempre messo article per "program documentation"
\usepackage[italian]{babel}
\usepackage[T1]{fontenc}
\usepackage[latin1]{inputenc}
\usepackage{titlesec}

\usepackage{emptypage}                     
% pagine vuote senza testatina e piede di pagina

%\usepackage{hyperref}                     
% collegamenti ipertestuali

\usepackage{fancyhdr}
% pacchetto per intestazione e pie pagina

\pagestyle{fancy}


% \\ indica interruzione di riga

% compilate 2 volte per documenti con indice

% {\em qui testo in corsivo}
% {\bfseries qui testo in grossetto}

%LISTE NUMERATE
%\begin{enumerate}
%\item primo
%\item secondo
%\item terzo
%\end{enumerate}

%LISTE PUNTATE
%\begin{itemize}
%\item primo
%\item secondo
%\item \dots
%\end{itemize}

%TABELLA
%\begin{tabular}{|c|c|c|}
%indica una tabella con 3 colonne e pos. testo centrale. La barra verticale ( | ) indica che vi e' una linea divisoria verticale tra le celle.
%\hline	linea separatrice orizzontale
%testo1& testo2& testo3\\
% & segna la fine del testo nella cella , \\ indica il fine della riga 

%GRAFICI
%\begin{figure}
%\includegraphics{filegrafico}
%comando per includere le immagini (controllare i formati)
%\caption{didascalia}
%\label{nome}
%\end{figure}


\lhead{SWEg Group}
\chead{}
\rhead{Nome Capitolo}
\lfoot{Analisi dei Requisiti}
\cfoot{}
\rfoot{\thepage}
\renewcommand{\headrulewidth}{0.4pt}
\renewcommand{\footrulewidth}{0.4pt}

\begin{document}

\title{ANALISI DEI REQUISITI}
\author{SWEg Group}
\date{}
\maketitle

\begin{tabular}{|l|c|c|c|}
\hline
{\textbf{Modifica}}&{\textbf{Nome}}&{\textbf{Data}}&{\textbf{Ver.}}\\
\hline
Creazione Raw Documento & Gianluca Crivellaro & 21/12/2016 & 0.0.1 \\
\hline
Modifica Raw & Gianluca Crivellaro & 22/12/2016 & 0.0.2 \\
\hline
Aggiunti Requisiti & Sebastiano Marchesini & 23/12/2016 & 0.0.3 \\
\hline
Aggiunti Requisiti Estesi Concordati & Gianluca Crivellaro & 27/12/2016 & 0.0.4 \\
\hline
Iniziata Stesura Documento (Introduzione) & Sebastiano Marchesini &  28/12/2016 & 0.1.0 \\
\hline
\end{tabular}
\newpage

\tableofcontents
%crea indice automaticamente
\thispagestyle{empty}

\newpage



\section{Introduzione}
\subsection{Scopo del Documento}
Tale documento ha lo scopo di studiare e modellare concettualmente il problema che si pone con APIM. Posizionando le componeti (o ambiti) a scopo di allocazione dei requisiti. Alcuni dei requisiti specificandoli con il diagramma dei casi d'uso.\\
Vi deve essere la certezza di non aver lasciato dimenticato nessuno tra i bisogno espliciti e i bisogni impliciti. Questo implica che non vi sia ambiguit� tra i requisiti.\\
Bisogna sempre tener conto di portare al massimo possibile la granularit� del problema, senza per� confonderlo e renderlo impossibile da verificare. Questo per rendere il requisito decidibile.\\
E' infine bene tener presente otto semplici qualit� di selezione dei requisiti:
\begin{itemize}
\item Non Ambigui
\item Corretti
\item Completi
\item Verificabili
\item Consistenti
\item Modificabili
\item Tracciabili
\item Ordinati per Rilevanza
\end{itemize}
\subsection{Scopo del Prodotto}
L'obbiettivo � creare un'infrastruttura che permetta la distribuzione digitale e la gestione dei diritti digitali di microservizi. Creati e importati da diversi utenti che possono interfacciarsi tra loro.\\
Viene usata per gestire e distribuire una vasta gamma di microservizi (alcuni esclusivi) e il loro relativo supporto. Tutte queste operazioni sono effettuate via Internet.\\
E' inoltre possibile il monitoraggio di ogni API grazie alle tecnologie fornite dal prodotto.\\ 
\subsection{Glossario}
Alla fine di evitare ambiguit� e mantenere la consistenza il Glossario � un documento unico e consultabile separatamente. \\
Un glossario � una raccolta di termini di un ambito specifico e circoscritto. In questo caso per raccogliere termini desueti e specialistici inerenti al progetto. 
\\
\subsection{Riferimenti Normativi}
\subsubsection{Normativi}
\item \textbf{Norme di Progetto}:	"Norme di Progetto v1.0.0".
\item \textbf{Capitolato d'appalto C1}:	API Market per microservizi	\href{www.math.unipd.it/~tullio/IS-1/2016/Progetto/C1.pdf}{www.math.unipd.it/~tullio/IS-1/2016/Progetto/C1.pdf}. 
\item \textbf{Verbali}:
\subsubsection{Informativi}
\item \textbf{Studio di Fattibilit�}: "Studio di Fattibilit� v.1.0.0".
\item \textbf{IEEE 830-1998}: Recommended Practice for Software Requirements Specifica- tions
\href{http://en.wikipedia.org/wiki/Software_requirements_specification}{http://en.wikipedia.org/wiki/Software_requirements_specification}.

\section{Descrizione Generale}
\subsection{Obbiettivi del prodotto}
L'obbiettivo primario del prodotto � di dare ad ogni utente la possibilit� di registrare il proprio microservizio in una piattaforma dedicata. In questo modo � possibile la vendita (o condivisione) con gli altri utenti della comunit� regolata da politiche di compravendita specifiche e flessibili a seconda dello scopo dell'API o del volere del tecnico. \\
L'obbiettivo � quindi quello di incentivare la programmazione a microservizi e, oltre a spingere i gruppi pi� piccoli nel progettare per il mercato virtuale, pensare sempre di pi� a delle migliori architetture flessibili invece che veri e propri programmi. Si vuole abbandonare i vecchi programmi monolite per entrare in una realt� fatta di sistemi divisi in moduli, la nuova sfida progettuale sar� quindi unire i vari microservizi (o API) per costruire un prodotto completo.
\subsection{Funzioni del prodotto}
SWEg Group si impegna in particolar modo alle sottoscritte funzioni del prodotto :
\item \textbf{Registrare le API di un microservizio}: dando la possibilit� di caricare un interfaccia e documentando la propria progettazione.
\item \textbf{Permetta di consultare le API}: con un sistema di ricerca designato e filtrato anche con dati tecnici . Anche se con minori funzionalit� anche un utente non registrato alla piattaforma pu� vagliare le varie API. Per ogni api sar� inoltre possibile un consulto dei suoi dati tecnici.
\item \textbf{Permetta di associare diverse API key}: cos� da regolare le politiche di scambio dei microservizi. Le API key sono lo strumento principale di collegamento tra la API e il suo utilizzatore. Grazie a queste l'infrastruttura potr� regolare le scadenze , l'utilizzo e procedimento oltre ad avere un ID univoco per la monitorizzazione. 
\item \textbf{Permetta di monitorare l'utilizzo delle API}: gi� accennato nei punti precedenti. Vogliamo che tale infrastruttura tenga conto di particolari dati tecnici di ogni API per renderle cos� misurabili in termini di efficacia ed efficienza. Oltre che a cos� avere un sistema automatizzato per il confronto tra i vari microservizi.
\item \textbf{Blocchi le chiamate di utenti in possesso di API key scadute e/o non valide}: � la sottolineatura di uno dei motivi di esistenza delle API key. Punto focale per la regolamentazione dello scambio � la possibilit� di acquisto in base
\subsection{Caratteristiche degli utenti}

\subsection{Piattaforma di esecuzione}

\subsection{Vincoli generali}


\end{document}