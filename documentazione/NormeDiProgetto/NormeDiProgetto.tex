% !TeX spellcheck = it_IT
\documentclass[12pt,a4paper,titlepage]{article}
%va sempre messo article per "program documentation"
\usepackage[italian]{babel}
\usepackage[T1]{fontenc}
\usepackage[latin1]{inputenc}
\usepackage{titlesec}
\usepackage{hyperref}
\usepackage[a4paper,top=2cm,bottom=2cm,left=3cm,right=3cm]{geometry}
\usepackage{soulutf8,color}


\usepackage{emptypage}                     
% pagine vuote senza testatina e piede di pagina

%\usepackage{hyperref}                     
% collegamenti ipertestuali

\usepackage{fancyhdr}
% pacchetto per intestazione e pie pagina

\pagestyle{fancy}


% \\ indica interruzione di riga

% compilate 2 volte per documenti con indice

% {\em qui testo in corsivo}
% {\bfseries qui testo in grossetto}

%LISTE NUMERATE
%\begin{enumerate}
%\item primo
%\item secondo
%\item terzo
%\end{enumerate}

%LISTE PUNTATE
%\begin{itemize}
%\item primo
%\item secondo
%\item \dots
%\end{itemize}

%TABELLA
%\begin{tabular}{|c|c|c|}
%indica una tabella con 3 colonne e pos. testo centrale. La barra verticale ( | ) indica che vi e' una linea divisoria verticale tra le celle.
%\hline	linea separatrice orizzontale
%testo1& testo2& testo3\\
% & segna la fine del testo nella cella , \\ indica il fine della riga 

%GRAFICI
%\begin{figure}
%\includegraphics{filegrafico}
%comando per includere le immagini (controllare i formati)
%\caption{didascalia}
%\label{nome}
%\end{figure}




\begin{document}

\title{NORME DI PROGETTO}
\author{SWEg Group}
\date{}
\maketitle

\lhead{SWEg Group}
\chead{}
\lfoot{Norme Di Progetto}
\cfoot{}
\rfoot{\thepage}
\renewcommand{\headrulewidth}{0.2pt}
\renewcommand{\footrulewidth}{0.2pt}
\rhead{Registro Modifiche}
\section{Registro Modifiche}
\small %rippicciolisce il testo
\begin{tabular}{|l|c|c|c|}
\hline
{\textbf{Modifica}}&{\textbf{Nome}}&{\textbf{Data}}&{\textbf{Ver.}}\\
\hline
Creazione Documento & Piergiorgio Danieli & 12/12/2016 & 0.0.1 \\
\hline
Aggiunti  & Sebastiano Marchesini & 12/12/2016 & 0.0.2 \\
\hline
Aggiunti & Piergiorgio Danieli & 13/12/2016 & 0.0.3 \\
\hline
Aggiunti & Sebastiano Marchesini & 14/12/2016 & 0.0.4 \\
\hline 
Aggiunti & Piergiorgio Danieli & 15/12/2016 & 0.0.5 \\
\hline
Aggiunti & Sebastiano Marchesini & 15/12/2016 & 0.0.6 \\
\hline
Stesura Iniziale & Sebastiano Marchesini & 30/12/2016 & 0.1.0 \\
\hline

\end{tabular}
\normalsize	%lo riporta a forma normale
\newpage

\tableofcontents
%crea indice automaticamente
\thispagestyle{empty}

\newpage



\rhead{Introduzione}
\section{Introduzione}
\subsection{Scopo del Documento}
Questo documento ha l'obiettivo di definire le regole che i membri del gruppo \textit{SWEG} seguiranno nello sviluppo del progetto.\\
Ogni componente del team e' tenuto a leggere e seguire le norme qui contenute per ottimizzare il lavoro, uniformare tutti i documenti e minimizzare gli errori.\\
In particolare verrano specificate le norme per:
\begin{itemize}
\item interazioni tra i membri del gruppo e componenti esterni
\item stesura dei documenti e relative convenzioni
\item ambiente di lavoro
\item stesura del codice.
\end{itemize}

\subsection{Glossario}
Al fine di evitare ambiguit� e ottimizzare la comprensione dei documenti, viene incluso un Glossario, nel quale saranno inseriti i termini tecnici, acronimi e parole che necessitano di essere chiarite.\\
Un glossario � una raccolta di termini di un ambito specifico e circoscritto. In questo caso per raccogliere termini desueti e specialistici inerenti al progetto. 

\subsection{Riferimenti}
%DA METTERE QUALCOSA QUI

\newpage

\rhead{Comunicazioni e Riunioni}
\section{Comunicazioni e Riunioni}

\subsection{Comunicazioni Interne ed Esterne}
\subsubsection{Interne}
Per le comunicazioni interne � stato creato un gruppo su \textit{Telegram}, un'app di messaggistica istantanea.
\subsubsection{Esterne}
Per le comunicazioni esterne � stata creata la casella di posta \textit{sweg.group@gmail.com}.\\ 
Questo pu� essere l'unico canale utilizzabile. Solo il \textit{Project Manager} pu� avere delle interazioni con l'esterno, sar� poi compito suo riferire il contenuto delle discussioni agli altri componenti del gruppo.

\subsection{Riunioni}
\subsubsection{Interne}
Le convocazioni alle riunioni sono notificate dal \textit{Project Manager} e confermate informalmente dai componenti del gruppo. Ogni membro � tenuto a partecipare a meno di una giustificazione valida. Una riunione e' fissata per discutere di argomenti di interesse generale di tutti i componenti. E' pero' possibile che ci sia la necessit� di effettuare delle riunioni alle quali non e' richiesta la presenza di tutti i membri ma solo di alcuni di essi; in questo caso il \textit{Project Manager} autorizza l'incontro e poi vi prenderanno parte solo le persone interessate, le quali dovranno poi informare gli altri componenti del gruppo delle decisioni prese.
\subsubsection{Esterne}
E' compito del \textit{Project Manager} fissare degli incontri con il proponete ed il committente.\\
Tutti i membri del gruppo ne devono essere informati. Ad ogni incontro verr� scelto un 
segretario il quale avr� il compito di redigere un verbale che verr� poi inviato a tutti
i componenti.\\
Ogni verbale redatto dovr� avere la seguente composizione: 
\begin{enumerate}
\item Data e ora
\item Luogo
\item Partecipanti interni
\item Partecipanti esterni
\item Argomenti trattati
\item Domande e risposte
\end{enumerate}
\newpage

\rhead{Documenti}
\section{Documenti}

\subsection{Template}
Per facilitare ed unificare la redazione della documentazione e' stato creato un 
template apposito in \Latex.

\subsubsection{Stile del testo}
\begin{itemize}
\item \textbf{Grossetto}: Deve essere applicato alle parole significative che devono essere messe in risalto;
\item \textbf{Corsivo}: va utilizzato per indicare termini in lingua inglese e nomi dei file;
\item \textbf{Maiuscolo}: va usato per scrivere gli acronimi;
\item \textbf{Latex}: ogni riferimento a \Latex va scritto utilizzando il comando \ Latex
\end{itemize}

\subsection{Convenzioni tipografiche}
%COMANDI IN LATEX
 
\subsection{Formati}
\begin{itemsize}
	\item \textbf{Date}: saranno espresse nel formato italiano gg/mm/aaaa dove:
	\begin{itemize}
		\item gg : indica il giorno, va scritto sempre con 2 cifre;
		\item mm : rappresenta il mese, va scritto sempre con 2 cifre;
		\item aaaa : rappresenta l'anno, va scritto sempre con 4 cifre. 
	\end{itemize}
	
	\item \textbf{Anni accademici}: si user� il formato aaaa1-aaaa2 dove:
	\begin{itemize}
		\item aaaa1: indica l'anno solare di inizio, va scritto sempre con 4 cifre;
		\item aaaa2: indica l'anno solare di fine, va scritto sempre con 4 cifre.
	\end{itemize}

	\item \textbf{Orari}: verranno espressi secondo lo standard \textit{ISO 8601} HH:MM dove:
	\begin{itemize}
		\item HH: indica le ore, va scritto sempre con 2 cifre;
		\item MM: indica i minuti, va scritto sempre con 2 cifre.
	\end{itemize}

	\item \textbf{URL}: collegamento ad un indirizzo web deve essere scritto con un font di colore blu;
	
	\item \textbf{Sigle}: i nomi dei documenti potranno essere sostituiti dalle rispettive sigle:
	\begin{itemize}
		\item AdR ad indicare il documento ?Analisi dei Requisiti?;
		\item PdP ad indicare il documento ?Piano di Progetto?;
		\item PdQ ad indicare il documento ?Piano di Qualifica?;
		\item NdP ad indicare il documento ?Norme di Progetto?;
		\item Gl ad indicare il documento ?Glossario?;
		\item ST ad indicare il documento ?Specifica Tecnica?;
		\item SdF ad indicare il documento ?Studio di Fattibilit�?.
	\end{itemize}
	
\subsection{Struttura del documento}
\subsubsection{Frontespizio}
Il frontespizio di ogni documento dovr� essere cos� strutturato:
\begin{itemize}
	\item Nome e logo del gruppo;
	\item Nome del progetto;
	\item Titolo del documento;
	\item Versione del documento;
	%AGGIUNGERE ALLA PRIMA PAGINA QUESTO
	\item Nome e cognome dei redattori del documento;
	\item Nome e cognome dei revisori del documento;
	\item Uso del documento;
	\item Destinatari del documento.
\end{itemize}
\subsubsection{Diario delle modifiche}
La seconda pagina deve essere una tabella contenente i cambiamenti che sono stati 
effettuati nel documento. Deve essere cos� strutturata:
\begin{itemize}
	\item Modifica: descrizione delle modifiche effettuate;
	\item Nome: nome e cognome dell'autore delle modifiche;
	\item Data: il giorno nel quale sono state apportate le modifiche;
	\item Versione: la versione del documento dopo la modifica;
\end{itemize}
La tabella � ordinata per data in ordine decrescente, in modo che la prima riga 
sia l'ultima modifica eseguita, e quindi corrisponda alla versione attuale del documento.
\subsubsection{Indice}
La pagine successiva al diario delle modifiche deve essere l'indice del documento.
Ogni documento tranne i verbali deve contenere l'indice che serve a dare una visione globale degli argomenti trattati.\\
L'indice � autogenerato con un comando di \LaTeX .
\subsubsection{Formattazione generale}
Tutte le altre pagine del documento dovranno rispettare la struttura di base che segue:
\begin{itemize}
	\item Logo del gruppo: sar� posizionato sempre in alto a sinistra;
	\item Sezione corrente: nell'intestazione deve comparire il numero ed il nome della sezione in cui ci si trova. Questa informazione sar� in alto a destra;
	\item Numero di pagina: compare a pi� di pagina, a destra.
\end{itemize} 
Ogni pagina deve rispettare i seguenti margini: 
\begin{itemize}
	\item Superiore: 2cm;
	\item Inferiore: 2cm;
	\item Destro: 1cm;
	\item Sinistro: 1cm;
\end{itemize}

\subsection{Classificazione documenti}
\subsubsection{Documenti formali}
Tutti i documenti che sono stati approvati dal \textit{Project Manager} sono da ritenersi formali.\\ 
Se un documento formale viene modificato, deve essere successivamente approvato nuovamente
dal \textit{Project Manager}. 
\subsubsection{Documenti informali}
Sono quei documenti che non sono stati approvati del Project Manager, poich� non lo necessitano e quindi sono ad esclusivo uso interno, o perch� sono ancora in fase di sviluppo e verranno 
approvati successivamente.

\subsection{Componenti grafiche}
\subsubsection{Tabelle}
Tutte le tabelle presenti nella documentazione necessitano di una didascalia che le descrive brevemente e che porta un numero progressivo in modo da identificarla univocamente.
\subsubsection{Immagini}
Le immagini come le tabelle dovranno essere accompagnate da una didascalia ed un numero progressivo che le identifica.

\subsection{Intestazione file di documentazione}
Ogni file di documentazione dovr� iniziare con la seguente intestazione:
\begin{itemize}
	\item \%FILE: nome del file;
	\item \%PERCORSO: /PercorsoDelFile/NomeDelDocumento/;
	\item \%DATA CREAZIONE: gg/mm/aa;
	\item \%AUTORE: SWEg;
	\item \%EMAIL: sweg.group@gmail.com;
\end{itemize}



\end{document}