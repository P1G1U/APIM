% !TeX spellcheck = it_IT
\documentclass[12pt,a4paper,titlepage]{article}
%va sempre messo article per "program documentation"
\usepackage[italian]{babel}
\usepackage[T1]{fontenc}
\usepackage[latin1]{inputenc}
\usepackage{titlesec}
\usepackage{hyperref}

\usepackage{emptypage}                     
% pagine vuote senza testatina e piede di pagina

%\usepackage{hyperref}                     
% collegamenti ipertestuali

\usepackage{fancyhdr}
% pacchetto per intestazione e pie pagina

\pagestyle{fancy}


% \\ indica interruzione di riga

% compilate 2 volte per documenti con indice

% {\em qui testo in corsivo}
% {\bfseries qui testo in grossetto}

%LISTE NUMERATE
%\begin{enumerate}
%\item primo
%\item secondo
%\item terzo
%\end{enumerate}

%LISTE PUNTATE
%\begin{itemize}
%\item primo
%\item secondo
%\item \dots
%\end{itemize}

%TABELLA
%\begin{tabular}{|c|c|c|}
%indica una tabella con 3 colonne e pos. testo centrale. La barra verticale ( | ) indica che vi e' una linea divisoria verticale tra le celle.
%\hline	linea separatrice orizzontale
%testo1& testo2& testo3\\
% & segna la fine del testo nella cella , \\ indica il fine della riga 

%GRAFICI
%\begin{figure}
%\includegraphics{filegrafico}
%comando per includere le immagini (controllare i formati)
%\caption{didascalia}
%\label{nome}
%\end{figure}




\begin{document}

\title{NORME DI PROGETTO}
\author{SWEg Group}
\date{}
\maketitle

\lhead{SWEg Group}
\chead{}
\lfoot{Norme Di Progetto}
\cfoot{}
\rfoot{\thepage}
\renewcommand{\headrulewidth}{0.2pt}
\renewcommand{\footrulewidth}{0.2pt}
\rhead{Registro Modifiche}
\section{Registro Modifiche}
\small %rippicciolisce il testo
\begin{tabular}{|l|c|c|c|}
\hline
{\textbf{Modifica}}&{\textbf{Nome}}&{\textbf{Data}}&{\textbf{Ver.}}\\
\hline
Creazione Documento & Piergiorgio Danieli & 12/12/2016 & 0.0.1 \\
\hline
Aggiunti  & Sebastiano Marchesini & 12/12/2016 & 0.0.2 \\
\hline
Aggiunti & Piergiorgio Danieli & 13/12/2016 & 0.0.3 \\
\hline
Aggiunti & Sebastiano Marchesini & 14/12/2016 & 0.0.4 \\
\hline 
Aggiunti & Piergiorgio Danieli & 15/12/2016 & 0.0.5 \\
\hline
Aggiunti & Sebastiano Marchesini & 15/12/2016 & 0.0.6 \\
\hline
Stesura Iniziale & Sebastiano Marchesini & 30/12/2016 & 0.1.0 \\
\hline

\end{tabular}
\normalsize	%lo riporta a forma normale
\newpage

\tableofcontents
%crea indice automaticamente
\thispagestyle{empty}

\newpage



\rhead{Introduzione}
\section{Introduzione}
\subsection{Scopo del Documento}
Questo documento ha l'obiettivo di definire le regole che i membri del gruppo \textit{SWEG} seguiranno nello sviluppo del progetto.\\
Ogni componente del team e' tenuto a leggere e seguire le norme qui contenute per ottimizzare il lavoro, uniformare tutti i documenti e minimizzare gli errori.\\
In particolare verrano specificate le norme per:
\begin{itemize}
\item interazioni tra i membri del gruppo e componenti esterni
\item stesura dei documenti e relative convenzioni
\item ambiente di lavoro
\item stesura del codice.
\end{itemize}

\subsection{Glossario}
Al fine di evitare ambiguit� e ottimizzare la comprensione dei documenti, viene incluso un Glossario, nel quale saranno inseriti i termini tecnici, acronimi e parole che necessitano di essere chiarite.\\
Un glossario � una raccolta di termini di un ambito specifico e circoscritto. In questo caso per raccogliere termini desueti e specialistici inerenti al progetto. 

\subsection{Riferimenti}
%DA METTERE QUALCOSA QUI

\newpage

\rhead{Comunicazioni e Riunioni}
\section{Comunicazioni e Riunioni}

\subsection{Comunicazioni Interne ed Esterne}
\subsubsection{Interne}
Per le comunicazioni interne � stato creato un gruppo su \textit{Telegram}, un'app di messaggistica istantanea.
\subsubsection{Esterne}
Per le comunicazioni esterne � stata creata la casella di posta \textit{sweg.group@gmail.com}.\\ 
Questo pu� essere l'unico canale utilizzabile. Solo il \textit{Project Manager} pu� avere delle interazioni con l'esterno, sar� poi compito suo riferire il contenuto delle discussioni agli altri componenti del gruppo.

\subsection{Riunioni}
\subsubsection{Interne}
Le convocazioni alle riunioni sono notificate dal \textit{Project Manager} e confermate informalmente dai componenti del gruppo. Ogni membro � tenuto a partecipare a meno di una giustificazione valida. Una riunione e' fissata per discutere di argomenti di interesse generale di tutti i componenti. E' pero' possibile che ci sia la necessit� di effettuare delle riunioni alle quali non e' richiesta la presenza di tutti i membri ma solo di alcuni di essi; in questo caso il \textit{Project Manager} autorizza l'incontro e poi vi prenderanno parte solo le persone interessate, le quali dovranno poi informare gli altri componenti del gruppo delle decisioni prese.
\subsubsection{Esterne}
E' compito del \textit{Project Manager} fissare degli incontri con il proponete ed il committente.\\
Tutti i membri del gruppo ne devono essere informati. Ad ogni incontro verr� scelto un 
segretario il quale avr� il compito di redigere un verbale che verr� poi inviato a tutti
i componenti.\\
Ogni verbale redatto dovr� avere la seguente composizione: 
\begin{enumerate}
\item Data e ora
\item Luogo
\item Partecipanti interni
\item Partecipanti esterni
\item Argomenti trattati
\item Domande e risposte

\newpage

\rhead{Documenti}
\section{Documenti}

\subsection{Template}
Per facilitare ed unificare la redazione della documentazione e' stato creato un 
template apposito in \Latex.

\subsubsection{Stile del testo}
\item \textbf{Grossetto}: Deve essere applicato alle parole significative che devono essere messe in risalto;
\item \textbf{Corsivo}: va utilizzato per indicare termini in lingua inglese e nomi dei file;
\item \textbf{Maiuscolo}: va usato per scrivere gli acronimi;
\item \textbf{Latex}: ogni riferimento a \Latex va scritto utilizzando il comando \ Latex
 

\subsection{Convenzioni tipografiche}

 
\subsection{Vincoli generali}
Per utilizzare le funzionalit� della piattaforma � obbligatorio avere una connessione internet.
%� necessario elencare i vincoli presenti sulle tecnologie richieste e sui sistemi operativi (anche browser) supportati. 
\newpage

\rhead{Casi d'uso}
\section{Caso d'uso}
L'analisi del capitolato, il dibattito tra gli Analisti e l'incontro con ItalianaSoftware ha portato alla creazione dei casi d'uso che seguono. Si � cercato anche di studiare piattaforme sociale e di scambio simili come \textit{Steam}, \textit{Google Play} e \textit{GitHub}.\\
Le fonti su cui si ricavano quindi i casi d'uso sono sia impliciti, derivanti dallo studio del dominio, sia espliciti, come il capitolato d'appalto.
Ogni caso d?uso ha un codice univoco gerarchico, nella forma: \\
UC[codice univoco del padre].[codice progressivo di livello] \\
Il codice progressivo puo? includere diversi livelli di gerarchia separati da un punto.\\
\subsection{Caso d'uso Generale: APIM}



\end{document}